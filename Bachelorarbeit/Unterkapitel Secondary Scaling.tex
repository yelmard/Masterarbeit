\subsection{Abschätzung der Sekundärteilchen}

%Quellen: Edgar, Julius, MKnichel Thesis. Analysis Note pp
Die im Unterkapitel \ref{sec:Spurselektion} beschriebenen Spurselektionen zielen unter Anderem darauf ab, die Kontamination der gemessenen Daten durch Sekundärteilchen zu entfernen. Trotzdem bleibt ein kleiner Anteil immer übrig. Dies fordert die Anwendung einer Korrektur, die den genannten Anteil aus den Daten extrahiert.\\
Wie schon angesprochen, werden den Sekundärteilchen Teilchen zugeordnet, die nicht unmittelbar mit der Kollision zusammenhängen, sondern die aus schwachen Zerfällen stammen, wie. z.B. aus Zerfälle von neutralen Kaonen $K^{0}$, von Lambda-Baryonen $\Lambda$ oder seltener von Myonen $\mu$. Neben solchen Fälle sind auch Teilchen, die aus Wechselwirkungen mit dem Detektormaterial resultieren, als sekundär definiert.\\
Zur Bestimmung des Anteils von Sekundärteilchen bedient man sich den $\mathrm{DCA}_{xy}$-Verteilungen. Das ist darauf zurückzuführen, dass die $\mathrm{DCA}_{xy}$ eines Primärteilchens sich von der eines Sekundärteilchens substantiell unterscheidet, denn der Ursprungspunkt der Teilchen der zweiten Gruppe befindet sich weiter entfernt vom Kollisionsvertex als der Ursprung der Primärteilchen. Es ist also erwartet, dass die jeweiligen $\mathrm{DCA}_{xy}$-Verteilungen verschiedenen Formen beschreiben. Dieser Unterschied wird besonders sichtlich bei sehr großen Werten von $\mathrm{DCA}_{xy}$.\\
Um allerdings die gesamte $\mathrm{DCA}_{xy}$-Verteilungen darstellen zu können, muss eine modifizierte Spurselektion angewendet werden. Die Selektionen $1$, $2$ und $10$ der Tabelle \ref{tab:Cuts} müssen nicht angewendet werden, da sie die $\mathrm{DCA}_{xy}$-Verteilungen direkt betreffen. Trotzdem muss man berücksichtigen, dass diese Änderungen an der Spurselektion die Funktionalität des ITS beeinträchtigen, welche neben anderen darin besteht, \textit{pile up} Ereignisse von der Messung auszuschließen. Als alternative Methode zum Ausschuss solcher Ereignisse wird die TOF verwendet.\\
Mit den Messdaten verfügt man über die $\mathrm{DCA}_{xy}$-Verteilung der gemessenen Teilchen, ohne zu unterscheiden, ob es sich um Primär- oder Sekundärteilchen handelt. Der vorhandene Bruchteil von Sekundärteilchen wird abgeschätzt, indem man unter Einsatz eines computergestützten Verfahrens diese Verteilung anhand einer linearen Kombination der mit den Monte-Carlo-Simulationen erhaltenen Verteilungen, die sogenannten \textit{templates}, parametrisiert. Aus der Monte-Carlo-Simulation erhält man eine $\mathrm{DCA}_{xy}$-Verteilung der Primärteilchen und eine der Sekundärteilchen.\\
Die $\mathrm{DCA}_{xy}$-Verteilungen werden in dieser Arbeit in drei verschiedenen $p_{\mathrm{T}}$-Bereichen dargestellt: $0,1 < p_{\mathrm{T}} < 0,5$ $\mathrm{GeV}/c$, $0,5 < p_{\mathrm{T}} < 1,0$ $\mathrm{GeV}/c$ und $1,0 < p_{\mathrm{T}} < 1,5$ $\mathrm{GeV}/c$. Die Analyse dieser $p_{\mathrm{T}}$-Intervalle genügt, um vorherzusagen, wie sich die Verteilung im gesamten $p_{\mathrm{T}}$-Bereich verhält. In der Abbildung \ref{RelativePtResolution} sind die $\mathrm{DCA}_{xy}$-Verteilungen geladener TEilchen in pp-Kollisionen bei $5,02$ TeV für diese drei Bereiche gezeigt. Die aus den Messdaten resultierende Kurve liegt oberhalb der den Primärteilchen entsprechenden Verteilung, die sich wiederum oberhalb der Verteilung für die Sekundärteilchen befindet. In allen Fällen lässt sich eine Verteilung von Ereignissen bei $\mathrm{DCA}_{xy} \approx 0$ cm, also bei der Umgebung des Nominalvertex. Insbesondere kann man erkennen, dass die Kurve der Sekundärteilchen eine breitere Anhäufung im Vergleich zu den anderen aufweist. Das bestätigt, dass die Ereignisse solcher Kurve nach Definition weiter vom Nominalvertex vorkommen. Die Abbildung \ref{RelativePtResolution} zeigt das Verhältnis von der Fit-Funktion zur $\mathrm{DCA}_{xy}$-Verteilung der Messdaten. Schließlich wird in der Abbildung \ref{ResultsFractionsDCA} die Anteile von Primärteilchen (rot) bzw. Sekundärteilchen (grün) in den drei $p_{\mathrm{T}}$-Bereiche gezeichnet. Dabei stellen die farbigen Symbolen die sich aus den Daten ergebenden Anteile dar, während die leeren Symbolen aus den Monte-Carlo-Simulationen stammen. Die gewünschte Korrektur des $p_{\mathrm{T}}$-Spektrums lässt sich aus dem Verhältnis von den Anteile der Daten zu den der Simulationen bestimmen.